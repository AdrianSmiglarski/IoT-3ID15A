\documentclass[a4paper,12pt]{article}
\usepackage{polski}
\usepackage[utf8]{inputenc}
\usepackage{color}
\usepackage{url}
\usepackage{hyperref}
\usepackage[normalem]{ulem}
\usepackage{latexsym}
\usepackage{enumerate}
\title{\huge Sprawozdanie lab1- IoT }
\date{2018-10-24}
\author{Adrian Śmiglarski\\ 3ID15A}
\frenchspacing
\begin{document}
\maketitle
\newpage
\section{Środowisko Tex Live}
\textbf{LaTeX} – oprogramowanie do zautomatyzowanego składu tekstu, a także związany z nim język znaczników, służący do formatowania dokumentów tekstowych i tekstowo-graficznych (na przykład: broszur, artykułów, książek, plakatów, prezentacji, a nawet stron HTML). W istocie LaTeX nie jest samodzielnym środowiskiem programistycznym. Jest to jedynie zestaw makr stanowiących nadbudowę dla systemu składu TEX, automatyzujących wiele czynności związanych z procesem poprawnego składania tekstu. Jednak, ze względu na dużą popularność LaTeX-a (w porównaniu z czystym TeX-em) nazwy te bywają używane zamiennie. Twórcą pierwszej wersji LaTeX-a był Leslie Lamport, a powstała ona w laboratorium badawczym firmy SRI International. Pierwowzorem był język Scribe.
\newline\newline \textbf{TeX Live} – dystrybucja TEX-a, opartego na nim oprogramowania (m.in. LaTeX, ConTeXt itp.) oraz oprogramowania pomocniczego (m.in. BibTeX).

\section{Formatowanie tekstu}

\subsection{Rodzaje czcionek}
\textit{Italics} \\
\textmd{Medium weight} \\
\textup{Upright} \\
\textbf{Boldface} \\
\texttt{Typewriter} \\ \\ 

\subsection{Rozmiar tekstu}
\normalsize Domyślna \newline \\
\large large  \newline \\
\Large Large  \newline \\
\huge huge  \newline \\ 
\small small  \newline \\
\tiny tiny \newline \\

\subsection{Pogrubienie, kursywa, podkreślenie}
\large \textbf{Czcionka pogrubiona} \\
\textit{Czcionka pochylona}\\ 
\uline{Tekst podkreślony pojedyńczą linią}\\
\uuline{Tekst podkreslony podwójną linią}

\subsection{ Kolorowanie tekstu}
{\color{red}  tekst czerwony} \\
{\color{green} tekst zielony}\\
{\color{blue}  tekst niebieski}\\
{\color{yellow} tekst zółty}
\subsection{Tabela}
\begin{table}[h!]
  \begin{center}
    \label{tab:table1}
    \begin{tabular}{l|c|r}
      \textbf{1} & \textbf{2} & \textbf{3}\\
      \hline
      a & d & g\\
      b & e & h\\
      c & f & i\\
    \end{tabular}
  \end{center}
\end{table}
\section{Bibliografia} \cite{1, 2, 3}
\bibliographystyle{plain}
\bibliography{mybib}{}
\end{document}